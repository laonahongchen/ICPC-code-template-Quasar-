%文档类型
\documentclass[landscape,twocolumn,a4paper]{article}
\setlength{\columnsep}{0.5cm}
\setlength{\columnseprule}{0.03cm}

%引用包裹
\usepackage{bm}
\usepackage{cmap}
\usepackage{ctex}
\usepackage{cite}
\usepackage{color}
\usepackage{float}
\usepackage{xeCJK}
\usepackage{amsthm}
\usepackage{amsmath}
\usepackage{amssymb}
\usepackage{setspace}
\usepackage{geometry}
\usepackage{hyperref}
\usepackage{enumerate}
\usepackage{indentfirst}
\usepackage[cache=false]{minted}
\usepackage{fontspec}
\usepackage{pdfpages}
\usepackage{fancyhdr}
\usepackage[table]{xcolor}
\usepackage{booktabs}
\usepackage{harpoon}
\usepackage{titlesec}
%代码高亮
%\geometry{margin=1in}
\geometry{a4paper,left=1.0cm,right=1.0cm,top=1.3cm,bottom=0.5cm}
\setmonofont{Consolas}
%字体设置
\setmainfont{Consolas}

\pagestyle{fancy}
\setlength{\headsep}{0.1cm}
\rhead{\thepage}
\lhead{上海交通大学 Shanghai Jiao Tong University}
\chead{Quasar}

\setlength{\columnsep}{30pt}

\hypersetup{hidelinks}
%\fontsize{5}{0}

\hypersetup{
	colorlinks=true,
	linkcolor=black
}

\titlespacing*{\section} {0pt}{0pt}{0pt}
\titlespacing*{\subsection} {0pt}{0pt}{0pt}
\titlespacing*{\subsubsection} {0pt}{-1pt}{-1pt}
\titleformat{\section}{\normalsize}{}{0pt}{}
\titleformat{\subsection}{\small}{}{0pt}{}
\titleformat{\subsubsection}{\small}{}{0pt}{}

\newcommand{\cppcode}[1]{
	\inputminted[mathescape,
	tabsize=2,
	style=xcode,
	baselinestretch=0.9,
	%linenos,
	%frame=single,
	autogobble,
	framesep=0.5mm,
	breakaftergroup=true,
	breakautoindent=true,
	breakbytoken=true,
	breaklines=true,
	numbersep=1mm,
	fontsize=\small
	]{cpp}{source/#1}
}
\newcommand{\javacode}[1]{
	\inputminted[mathescape,
	tabsize=2,
	%linenos,
	%frame=single,
	framesep=2mm,
	autogobble,
	numbersep=1mm,
	baselinestretch=0.9,
	breakaftergroup=true,
	breakautoindent=true,
	breakbytoken=true,
	breaklines=true,
	fontsize=\small
	]{java}{source/#1}
}

\newcommand{\vimcode}[1]{
	\inputminted[mathescape,
	tabsize=2,
	%linenos,
	%frame=single,
	framesep=2mm,
	breakaftergroup=true,
	breakautoindent=true,
	breakbytoken=true,
	breaklines=true,
	fontsize=\small
	]{vim}{source/#1}
}
\begin{document}
\begin{titlepage}
	\title{Standard Code Library}
	\author{QuasaR}
%	\date{\today}
	\maketitle
\end{titlepage}
\tableofcontents
\small


\section{数学}

	\subsection{快速求逆元(内含exgcd)}
	\indent 使用条件:$x \in [0, mod)$并且$x$与$mod$互质
	\cppcode{number-theory/extended-euclid.cpp}
	\cppcode{number-theory/inverse.cpp}


	\subsection{中国剩余定理}
	返回结果:$x \equiv r_i (mod \ p_i) \ (0 \leq i < n)$
	\cppcode{number-theory/chinese-remainder-theorem.cpp}
	\cppcode{number-theory/china2.cpp}

	\subsection{小步大步}
	返回结果:$a^x=b \ (mod \ p)$
	\indent 使用条件:$p$为质数
	\cppcode{number-theory/exBSGS.cpp}

	\subsection{Miller Rabin 素数测试}
	\cppcode{number-theory/miller-rabin.cpp}

	\subsection{Pollard Rho 大数分解}
	\cppcode{number-theory/pollard-rho.cpp}

	\subsection{NTT}
	\cppcode{numerical-algorithm/ntt.cpp}

	\subsection{原根}
	\cppcode{number-theory/primeroot.cpp}

	\subsection{线性递推}
	\cppcode{number-theory/linear-recurrence.cpp}

	\subsection{直线下整点个数}
	返回结果:$\sum_{0 \leq i < n} \lfloor \frac{a + b \cdot i}{m} \rfloor$
	\indent 使用条件:$n, m > 0$,$a, b \geq 0$
	\indent 时间复杂度:$\mathcal{O}(n log n)$
	\cppcode{number-theory/lattice-count.cpp}

	\subsection{高斯消元}
	\cppcode{numerical-algorithm/Gauss.cpp}

	\subsection{FFT}
	\cppcode{numerical-algorithm/fft_gls.cpp}

	\subsection{1e9+7 FFT}
	\cppcode{numerical-algorithm/fft.cpp}

	\subsection{FWT}
	\cppcode{number-theory/fwt.cpp}

	\subsection{自适应辛普森}
	\cppcode{numerical-algorithm/adaptive-simpson.cpp}

	\subsection{多项式求根}
	\cppcode{numerical-algorithm/polyroot.cpp}


\section{数据结构}

	\subsection{lct}
	\cppcode{Data-Structure/LCT.cpp}


	\subsection{树上莫队}
	\cppcode{Data-Structure/modui_tree_ver.cpp}

	\subsection{树状数组kth}
	\cppcode{Data-Structure/fenwicktree.cpp}

	\subsection{虚树}
	\cppcode{Data-Structure/virtualtree.cpp}

\section{图论}

	\subsection{点双连通分量(lyx)}
	\cppcode{Graph-Algorithm/bcc.cpp}

	\subsection{Hopcoft-Karp求最大匹配}
	\cppcode{Graph-Algorithm/Maximum-Matching-Hopcroft-Karp.cpp}

	\subsection{KM带权匹配}
	\noindent \textbf{注意事项:}最小权完美匹配,复杂度为$\mathcal{O}(|V|^3)$。
	\cppcode{Graph-Algorithm/Maximum-Weight-Matching.cpp}

	\subsection{zkw费用流}
	\cppcode{Graph-Algorithm/Minimum-Cost-Maxflow-ZKW.cpp}

	\subsection{2-SAT问题}
	\cppcode{Graph-Algorithm/Two-Satisfiability.cpp}

	\subsection{有根树的同构}
	\cppcode{Graph-Algorithm/Rooted-Tree-Isomorphism.cpp}

	\subsection{Dominator Tree}
	\cppcode{Graph-Algorithm/dominator.cpp}

	\subsection{无向图最小割}
	\cppcode{Graph-Algorithm/Minimum-Cut-Stoer-Wagner.cpp}

	\subsection{带花树}
	\cppcode{Graph-Algorithm/Maximum-Matching-Blossom.cpp}

\section{字符串}
\subsection{KMP}
\cppcode{String/kmp.cpp}

\subsection{EXKMP}
\cppcode{String/exkmp.cpp}

\subsection{AC自动机}
\cppcode{String/acauto.cpp}

\subsection{SAM}
\cppcode{String/sam.cpp}

\subsection{后缀数组}
\cppcode{String/sa.cpp}

\subsection{回文自动机}
\cppcode{String/pam.cpp}

\subsection{Manacher}
\cppcode{String/Manacher.cpp}
\subsection{循环串的最小表示}
\noindent \textbf{注意事项:}0-Based算法,请注意下标。
\cppcode{temp_ypm/minimumpresentation.cpp}


\section{计算几何}

	\subsection{二维几何}
%	\cppcode{Geometry/8 Geo2D_simple.cpp}
	\cppcode{Geometry/Geo2D.cpp}
%	\subsection{凸包}
%	\cppcode{Geometry/convex.cpp}

	\subsection{阿波罗尼茨圆}
	\cppcode{Geometry/3 abolonis.cpp}

	\subsection{三角形与圆交}
	\cppcode{Geometry/areaCT.cpp}

	\subsection{圆并}
	\cppcode{Geometry/CircleArea.cpp}

	\subsection{整数半平面交}
	\cppcode{Geometry/HPI_integer.cpp}
	\subsection{半平面交}
	\cppcode{Geometry/HalfPlaneIntersection.cpp}

	\subsection{三角形}
	\cppcode{Geometry/Triangle.cpp}

	\subsection{经纬度求球面最短距离}
	\cppcode{Geometry/12.cpp}

	\subsection{长方体表面两点最短距离}
	\cppcode{Geometry/13.cpp}

	\subsection{点到凸包切线}
	\cppcode{Geometry/14.cpp}

	\subsection{直线与凸包的交点}
	\cppcode{Geometry/15.cpp}

	\subsection{平面最近点对}
	\cppcode{Geometry/nearest.cpp}

	\subsection{三维几何}
	\cppcode{Geometry/3DGeo.cpp}


\section{其他}

	\subsection{最小树形图}
	\cppcode{miscellany/mintreegraph.cpp}

	\subsection{DLX}
	\cppcode{miscellany/DLX.cpp}

	\subsection{某年某月某日是星期几}
	\cppcode{miscellany/what-day-is-today.cpp}

	\subsection{枚举大小为$k$的子集}
	使用条件:$k > 0$
	\cppcode{miscellany/subset-of-size-k.cpp}

	\subsection{环状最长公共子串}
	\cppcode{miscellany/cyclic-longest-common-string.cpp}

	\subsection{LLMOD STL内存清空 开栈}
	\cppcode{miscellany/LLMOD.cpp}
	\cppcode{Hint/STL-memory-release.cpp}
	\cppcode{Hint/openstack.cpp}

	\subsection{vimrc}
	\vimcode{Hint/vimrc}

	\subsection{上下界网络流}

	\subsubsection*{无源汇的上下界可行流}
	建立超级源点$S^*$和超级汇点$T^*$,对于原图每条边$(u,v)$在新网络中连如下三条边:$S^* \rightarrow v$,容量为$B(u,v)$;$u \rightarrow T^*$,容量为$B(u,v)$;$u \rightarrow v$,容量为$C(u,v) - B(u,v)$。最后求新网络的最大流,判断从超级源点$S^*$出发的边是否都满流即可,边$(u,v)$的最终解中的实际流量为$G(u,v)+B(u,v)$。

	\subsubsection*{有源汇的上下界可行流}
	从汇点$T$到源点$S$连一条上界为$\infty$,下界为$0$的边。按照\textbf{无源汇的上下界可行流}一样做即可,流量即为$T \rightarrow S$边上的流量。

	\subsubsection*{有源汇的上下界最大流}
	\begin{enumerate}
		\item 在\textbf{有源汇的上下界可行流}中,从汇点$T$到源点$S$的边改为连一条上界为$\infty$,下届为$x$的边。$x$满足二分性质,找到最大的$x$使得新网络存在\textbf{无源汇的上下界可行流}即为原图的最大流。
		\item 从汇点$T$到源点$S$连一条上界为$\infty$,下界为$0$的边,变成无源汇的网络。按照\textbf{无源汇的上下界可行流}的方法,建立超级源点$S^*$和超级汇点$T^*$,求一遍$S^* \rightarrow T^*$的最大流,再将从汇点$T$到源点$S$的这条边拆掉,求一次$S \rightarrow T$的最大流即可。
	\end{enumerate}

	\subsubsection*{有源汇的上下界最小流}
	\begin{enumerate}
		\item 在\textbf{有源汇的上下界可行流}中,从汇点$T$到源点$S$的边改为连一条上界为$x$,下界为$0$的边。$x$满足二分性质,找到最小的$x$使得新网络存在\textbf{无源汇的上下界可行流}即为原图的最小流。
		\item 按照\textbf{无源汇的上下界可行流}的方法,建立超级源点$S^*$与超级汇点$T^*$,求一遍$S^* \rightarrow T^*$的最大流,但是注意这一次不加上汇点$T$到源点$S$的这条边,即不使之改为无源汇的网络去求解。求完后,再加上那条汇点$T$到源点$S$上界$\infty$的边。因为这条边下界为$0$,所以$S^*$,$T^*$无影响,再直接求一次$S^* \rightarrow T^*$的最大流。若超级源点$S^*$出发的边全部满流,则$T \rightarrow S$边上的流量即为原图的最小流,否则无解。
	\end{enumerate}

	\subsection{上下界费用流}
	\noindent 设汇$t$,源$s$,超级源$S$,超级汇$T$,本质是每条边的下界为1,上界为MAX,跑一遍有源汇的上下界最小费用最小流。(因为上界无穷大,所以只要满足所有下界的最小费用最小流)

	\begin{enumerate}
		\item 对每个点$x$:从$x$到$t$连一条费用为0,流量为MAX的边,表示可以任意停止当前的剧情(接下来的剧情从更优的路径去走,画个样例就知道了)
		\item 对于每一条边权为z的边x->y:

		\begin{itemize}
			\item 从S到y连一条流量为1,费用为z的边,代表这条边至少要被走一次。
			\item 从x到y连一条流量为MAX,费用为z的边,代表这条边除了至少走的一次之外还可以随便走。
			\item 从x到T连一条流量为1,费用为0的边。(注意是每一条x->y的边都连,或者你可以记下x的出边数Kx,连一次流量为Kx,费用为0的边)。

		\end{itemize}
	\end{enumerate}
	建完图后从S到T跑一遍费用流,即可。(当前跑出来的就是满足上下界的最小费用最小流了)

	\subsection{Bernoulli数}
	\begin{enumerate}
	\item 初始化:$B_0(n) = 1$
	\item 递推公式:
	%\[B_m(n) = n^m - \sum_{k = 0}^{m - 1}\binom{m}{k} \frac{B_k(n)}{m - k + 1}\]
	$B_m(n) = n^m - \sum_{k = 0}^{m - 1}\binom{m}{k} \frac{B_k(n)}{m - k + 1}$
	\item 应用:
	%\[\sum_{k = 1}^{n} k^m = \frac{1}{m + 1}\sum_{k = 0}^{m}\binom{m + 1}{k}n^{m + 1 - k}\]
	$\sum_{k = 1}^{n} k^m = \frac{1}{m + 1}\sum_{k = 0}^{m}\binom{m + 1}{k}n^{m + 1 - k}$
\end{enumerate}


	\subsection{Java Hints}
	\javacode{Hint/Java-Hints.java}

	\subsection{String Hints}
	1.多个串的最长公共子串
	(i)sa:二分答案分组。
	(ii)sam:对第一个串建立sam,其他匹配,对每个节点维护到达该节点所能匹配上的最大长度,按照拓扑倒序用每个节点去更新parent节点。
	2.重复次数最多的连续重复子串:
	枚举长度L,求长度为L的子串最多的连续次数。枚举位置$0,L,2L,3L……s[L*i]$和$s[L*(i+1)]$往前和往后能匹配的总长度为k,那么这里连续出现了k/L+1次。
	3.统计子串数目问题
	(1)本质不同的子串个数:
	(i)$n-sa[i]-height[i](0-base)$
	(ii)$T[x].len-T[T[x].root].len$。
	(iii)$siz[x]=∑siz[T[x].nx[i]]+1$。
	(2)长度不小于k的公共子串(S和T)的个数(位置不同算多次)
	(i)sa:后缀分组,用单调栈维护T后缀和前面所有S后缀的lcp之和,S后缀和前面的所有T后缀类似。
	(ii)sam:构建S的sam,T匹配。
	$f[x] += (T[x].len - max(T[T[x].root].len + 1, k) + 1) * siz[x]$;
	$ans += f[T[p].root] + (len - max(T[T[p].root].len + 1, k) + 1) * siz[p]$;
	4.出现问题
	(1)多次出现算多次:
	多次询问串a在串b中出现了多少次。
	考虑单次询问,将串b的每个前缀在fail树中对应的节点到根的路径+1,求串a在fail树中对应的节点被标记了多少次。
	对于多次询问,只需要将询问按照b在打字机串中出现的顺序排序更新即可。
	(2)多次出现算一次:
	有两个字符串集合A,B,询问A中的每个串a被B中的多少个串b包含,询问B中的每个串b包含A中的多少个串a。
	对集合A中的串建立自动机:
	每个a的答案是该串对应的节点的子树中出现了多少个b串的前缀(只需要在危险节点处标记),数颜色问题。
	每个b的答案是该串的所有前缀对应的节点到根节点的路径上出现了多少个不同的串a,树链的并。
	求出siz[x]表示节点x以及它的所有后缀中有多少个串a,将b所有前缀对应的节点按照dfs序统计即可。
	5.找第K小的子串$siz[x]=∑siz[T[x].nx[i]]+1/siz[x]=∑siz[T[x].nx[i]]+right$集合的大小。

\section{数学}

\subsection{常用数学公式}

	\subsubsection*{求和公式}

	\begin{enumerate}
		\item $\sum_{k=1}^{n}(2k-1)^2 = \frac{n(4n^2-1)}{3}	$
		\item $\sum_{k=1}^{n}k^3 = [\frac{n(n+1)}{2}]^2	$
		\item $\sum_{k=1}^{n}(2k-1)^3 = n^2(2n^2-1)	$
		\item $\sum_{k=1}^{n}k^4 = \frac{n(n+1)(2n+1)(3n^2+3n-1)}{30}  $
		\item $\sum_{k=1}^{n}k^5 = \frac{n^2(n+1)^2(2n^2+2n-1)}{12}	$
		\item $\sum_{k=1}^{n}k(k+1) = \frac{n(n+1)(n+2)}{3}	$
		\item $\sum_{k=1}^{n}k(k+1)(k+2) = \frac{n(n+1)(n+2)(n+3)}{4} $
		\item $\sum_{k=1}^{n}k(k+1)(k+2)(k+3) = \frac{n(n+1)(n+2)(n+3)(n+4)}{5} $
	\end{enumerate}

	\subsubsection*{斐波那契数列}

	\begin{enumerate}
		\item $fib_0=0, fib_1=1, fib_n=fib_{n-1}+fib_{n-2}$
		\item $fib_{n+2} \cdot fib_n-fib_{n+1}^2=(-1)^{n+1}$
		\item $fib_{-n}=(-1)^{n-1}fib_n$
		\item $fib_{n+k}=fib_k \cdot fib_{n+1}+fib_{k-1} \cdot fib_n$
		\item $gcd(fib_m, fib_n)=fib_{gcd(m, n)}$
		\item $fib_m|fib_n^2\Leftrightarrow nfib_n|m$
	\end{enumerate}

	\subsubsection*{错排公式}

	\begin{enumerate}
		\item $D_n = (n-1)(D_{n-2}-D_{n-1})$
		\item $D_n = n! \cdot (1-\frac{1}{1!}+\frac{1}{2!}-\frac{1}{3!}+\ldots+\frac{(-1)^n}{n!})$
	\end{enumerate}

	\subsubsection*{莫比乌斯函数}
	\iffalse
	$\mu(n) = \begin{cases}
	1 & \text{若}n=1\\
	(-1)^k & \text{若}n\text{无平方数因子,且}n = p_1p_2\dots p_k\\
	0 & \text{若}n\text{有大于}1\text{的平方数因数}
	\end{cases}$
	$\sum_{d|n}{\mu(d)} = \begin{cases}
	1 & \text{若}n=1\\
	0 & \text{其他情况}
	\end{cases}$
	\fi
	$g(n) = \sum_{d|n}{f(d)} \Leftrightarrow f(n) = \sum_{d|n}{\mu(d)g(\frac{n}{d})}$
	$g(x) = \sum_{n=1}^{[x]}f(\frac{x}{n}) \Leftrightarrow f(x) = \sum_{n=1}^{[x]}{\mu(n)g(\frac{x}{n})}$

	\subsubsection*{伯恩赛德引理}
	设$G$是一个有限群,作用在集合$X$上。对每个$g$属于$G$,令$X^g$表示$X$中在$g$作用下的不动元素,轨道数(记作$|X/G|$)由如下公式给出:
	$|X/G| = \frac{1}{|G|}\sum_{g \in G}|X^g|.\,$

	\subsubsection*{五边形数定理}

	设$p(n)$是$n$的拆分数,有$p(n) = \sum_{k \in \mathbb{Z} \setminus \{0\}} (-1)^{k - 1} p\left(n - \frac{k(3k - 1)}{2}\right)$

	\subsubsection*{树的计数}

	\begin{enumerate}
		\item 有根树计数:$n+1$个结点的有根树的个数为
		$a_{n+1} = \frac{\sum_{j=1}^{n}{j \cdot a_j \cdot{S_{n, j}}}}{n}$
		其中,
		$S_{n, j} = \sum_{i=1}^{n/j}{a_{n+1-ij}} = S_{n-j, j} + a_{n+1-j}$
		\item 无根树计数:当$n$为奇数时,$n$个结点的无根树的个数为
		$a_n-\sum_{i=1}^{n/2}{a_ia_{n-i}}$
		当$n$为偶数时,$n$个结点的无根树的个数为
		$a_n-\sum_{i=1}^{n/2}{a_ia_{n-i}}+\frac{1}{2}a_{\frac{n}{2}}(a_{\frac{n}{2}}+1)$
		\item $n$个结点的完全图的生成树个数为
		$n^{n-2}$
		\item 矩阵-树定理:图$G$由$n$个结点构成,设$\bm{A}[G]$为图$G$的邻接矩阵、$\bm{D}[G]$为图$G$的度数矩阵,则图$G$的不同生成树的个数为$\bm{C}[G] = \bm{D}[G] - \bm{A}[G]$的任意一个$n-1$阶主子式的行列式值。
	\end{enumerate}

	\subsubsection*{欧拉公式}

	平面图的顶点个数、边数和面的个数有如下关系:
	$V - E + F = C+ 1$
	\indent 其中,$V$是顶点的数目,$E$是边的数目,$F$是面的数目,$C$是组成图形的连通部分的数目。当图是单连通图的时候,公式简化为:
	$V - E + F = 2$

	\subsubsection*{皮克定理}

	给定顶点坐标均是整点(或正方形格点)的简单多边形,其面积$A$和内部格点数目$i$、边上格点数目$b$的关系:
	$A = i + \frac{b}{2} - 1$

	\subsubsection*{牛顿恒等式}

	设$\prod_{i = 1}^n{(x - x_i)} = a_n + a_{n - 1} x + \dots + a_1 x^{n - 1} + a_0 x^n$
	$p_k = \sum_{i = 1}^n{x_i^k}$
	则$a_0 p_k + a_1 p_{k - 1} + \cdots + a_{k - 1} p_1 + k a_k = 0$

	特别地,对于$|\bm{A} - \lambda \bm{E}| = (-1)^n(a_n + a_{n - 1} \lambda + \cdots + a_1 \lambda^{n - 1} + a_0 \lambda^n)$
	有$p_k = Tr(\bm{A}^k)$


	\subsection{平面几何公式}

	\subsubsection*{三角形}

	\begin{enumerate}
		\item 面积
		$S=\frac{a \cdot H_a}{2}=\frac{ab \cdot sinC}{2}=\sqrt{p(p-a)(p-b)(p-c)}$
		\item 中线
		$M_a=\frac{\sqrt{2(b^2+c^2)-a^2}}{2}=\frac{\sqrt{b^2+c^2+2bc \cdot cosA}}{2}$
		\item 角平分线
		$T_a=\frac{\sqrt{bc \cdot [(b+c)^2-a^2]}}{b+c}=\frac{2bc}{b+c}cos\frac{A}{2}$
		\item 高线
		$H_a=bsinC=csinB=\sqrt{b^2-(\frac{a^2+b^2-c^2}{2a})^2}$
		\item 内切圆半径
		\begin{align*}
		r&=\frac{S}{p}=\frac{arcsin\frac{B}{2} \cdot sin\frac{C}{2}}{sin\frac{B+C}{2}}=4R \cdot sin\frac{A}{2}sin\frac{B}{2}sin\frac{C}{2}\\
		&=\sqrt{\frac{(p-a)(p-b)(p-c)}{p}}=p \cdot tan\frac{A}{2}tan\frac{B}{2}tan\frac{C}{2}
		\end{align*}
		\item 外接圆半径
		$R=\frac{abc}{4S}=\frac{a}{2sinA}=\frac{b}{2sinB}=\frac{c}{2sinC}$
	\end{enumerate}

	\subsubsection*{四边形}

	$D_1, D_2$为对角线,$M$对角线中点连线,$A$为对角线夹角,$p$为半周长
	\begin{enumerate}
		\item $a^2+b^2+c^2+d^2=D_1^2+D_2^2+4M^2$
		\item $S=\frac{1}{2}D_1D_2sinA$
		\item 对于圆内接四边形
		$ac+bd=D_1D_2$
		\item 对于圆内接四边形
		$S=\sqrt{(p-a)(p-b)(p-c)(p-d)}$
	\end{enumerate}

	\subsubsection*{正$n$边形}

	$R$为外接圆半径,$r$为内切圆半径
	\begin{enumerate}
		\item 中心角
		$A=\frac{2\pi}{n}$
		\item 内角
		$C=\frac{n-2}{n}\pi$
		\item 边长
		$a=2\sqrt{R^2-r^2}=2R \cdot sin\frac{A}{2}=2r \cdot tan\frac{A}{2}$
		\item 面积
		$S=\frac{nar}{2}=nr^2 \cdot tan\frac{A}{2}=\frac{nR^2}{2} \cdot sinA=\frac{na^2}{4 \cdot tan\frac{A}{2}}$
	\end{enumerate}

	\subsubsection*{圆}

	\begin{enumerate}
		\item 弧长
		$l=rA$
		\item 弦长
		$a=2\sqrt{2hr-h^2}=2r\cdot sin\frac{A}{2}$
		\item 弓形高
		$h=r-\sqrt{r^2-\frac{a^2}{4}}=r(1-cos\frac{A}{2})=\frac{1}{2} \cdot arctan\frac{A}{4}$
		\item 扇形面积
		$S_1=\frac{rl}{2}=\frac{r^2A}{2}$
		\item 弓形面积
		$S_2=\frac{rl-a(r-h)}{2}=\frac{r^2}{2}(A-sinA)$
	\end{enumerate}

	\subsubsection*{棱柱}

	\begin{enumerate}
		\item 体积
		$V=Ah$
		$A$为底面积,$h$为高
		\item 侧面积
		$S=lp$
		$l$为棱长,$p$为直截面周长
		\item 全面积
		$T=S+2A$
	\end{enumerate}

	\subsubsection*{棱锥}

	\begin{enumerate}
		\item 体积
		$V=Ah$
		$A$为底面积,$h$为高
		\item 正棱锥侧面积
		$S=lp$
		$l$为棱长,$p$为直截面周长
		\item 正棱锥全面积
		$T=S+2A$
	\end{enumerate}

	\subsubsection*{棱台}

	\begin{enumerate}
		\item 体积
		$V=(A_1+A_2+\sqrt{A_1A_2}) \cdot \frac{h}{3}$
		$A_1,A_2$为上下底面积,$h$为高    正棱台侧面积
		$S=\frac{p_1+p_2}{2}l$
		$p_1,p_2$为上下底面周长,$l$为斜高
		\item 正棱台全面积
		$T=S+A_1+A_2$
	\end{enumerate}

	\subsubsection*{圆柱}

	\begin{enumerate}
		\item 侧面积
		$S=2\pi rh$
		\item 全面积
		$T=2\pi r(h+r)$
		\item 体积
		$V=\pi r^2h$
	\end{enumerate}

	\subsubsection*{圆锥}

	\begin{enumerate}
		\item 母线
		$l=\sqrt{h^2+r^2}$
		\item 侧面积
		$S=\pi rl$    全面积
		$T=\pi r(l+r)$
		\item 体积
		$V=\frac{\pi}{3} r^2h$
	\end{enumerate}

	\subsubsection*{圆台}

	\begin{enumerate}
		\item 母线
		$l=\sqrt{h^2+(r_1-r_2)^2}$
		\item 侧面积
		$S=\pi(r_1+r_2)l$    全面积
		$T=\pi r_1(l+r_1)+\pi r_2(l+r_2)$
		\item 体积
		$V=\frac{\pi}{3}(r_1^2+r_2^2+r_1r_2)h$
	\end{enumerate}

	\subsubsection*{球台}

	\begin{enumerate}
		\item 侧面积
		$S=2\pi rh$   全面积
		$T=\pi(2rh+r_1^2+r_2^2)$
		\item 体积
		$V=\frac{\pi h[3(r_1^2+r_2^2)+h^2]}{6}$
	\end{enumerate}

	\subsubsection*{球扇形}

	\begin{enumerate}
		\item 全面积
		$T=\pi r(2h+r_0)$
		$h$为球冠高,$r_0$为球冠底面半径
		\item 体积
		$V=\frac{2}{3}\pi r^2h$
	\end{enumerate}

	\subsection{积分表}
	\begin{footnotesize}
\mbox{\vbox to 11pt{  \hbox{$
\int \frac{1}{1+x^2}dx = \tan^{-1}x
$}  }}
\\
\mbox{\vbox to 11pt{  \hbox{$
\int \frac{1}{a^2+x^2}dx = \frac{1}{a}\tan^{-1}\frac{x}{a}
$}  }}
\\
\mbox{\vbox to 11pt{  \hbox{$
\int \frac{x}{a^2+x^2}dx = \frac{1}{2}\ln|a^2+x^2|
$}  }}
\\
\mbox{\vbox to 11pt{  \hbox{$
\int \frac{x^2}{a^2+x^2}dx = x-a\tan^{-1}\frac{x}{a}
$}  }}
\\
\mbox{\vbox to 11pt{  \hbox{$
\int\sqrt{x^2 \pm a^2} dx  = \frac{1}{2}x\sqrt{x^2\pm a^2} 
%\nonumber \\ 
\pm\frac{1}{2}a^2 \ln \left | x + \sqrt{x^2\pm a^2} \right | 
$}  }}
\\
\mbox{\vbox to 11pt{  \hbox{$
\int  \sqrt{a^2 - x^2} dx  = \frac{1}{2} x \sqrt{a^2-x^2} 
%\nonumber \\  
+\frac{1}{2}a^2\tan^{-1}\frac{x}{\sqrt{a^2-x^2}}
$}  }}
\\
\mbox{\vbox to 11pt{  \hbox{$
\int \frac{x^2}{\sqrt{x^2 \pm a^2}} dx  = \frac{1}{2}x\sqrt{x^2 \pm a^2}
%\nonumber \\  
\mp \frac{1}{2}a^2 \ln \left| x + \sqrt{x^2\pm a^2} \right | 
$}  }}
\\
\mbox{\vbox to 11pt{  \hbox{$
\int \frac{1}{\sqrt{x^2 \pm a^2}} dx = \ln \left | x + \sqrt{x^2 \pm a^2} \right | 
$}  }}
\\
\mbox{\vbox to 11pt{  \hbox{$
\int \frac{1}{\sqrt{a^2 - x^2}} dx = \sin^{-1}\frac{x}{a} 
$}  }}
\\
\mbox{\vbox to 11pt{  \hbox{$
\int \frac{x}{\sqrt{x^2\pm a^2}}dx = \sqrt{x^2 \pm a^2} 
$}  }}
\\
\mbox{\vbox to 11pt{  \hbox{$
\int \frac{x}{\sqrt{a^2-x^2}}dx = -\sqrt{a^2-x^2} 
$}  }}
\\
\mbox{\vbox to 11pt{  \hbox{$
\int  \sqrt{a x^2 + b x + c} dx = 
\frac{b+2ax}{4a}\sqrt{ax^2+bx+c}
\nonumber \\  
+
\frac{4ac-b^2}{8a^{3/2}}\ln \left| 2ax + b + 2\sqrt{a(ax^2+bx^+c)}\right |
$}  }}
\\
\mbox{\vbox to 11pt{  \hbox{$
\int x^n e^{ax}\hspace{1pt}\text{d}x = \dfrac{x^n e^{ax}}{a} - 
\dfrac{n}{a}\int x^{n-1}e^{ax}\hspace{1pt}\text{d}x
$}  }} 
\\
\mbox{\vbox to 11pt{  \hbox{$
\int \sin^2 ax dx = \frac{x}{2} - \frac{1} {4a} \sin{2ax}
$}  }}
\\
\mbox{\vbox to 11pt{  \hbox{$
\int \sin^3 ax dx = -\frac{3 \cos ax}{4a} + \frac{\cos 3ax} {12a} 
$}  }}
\\
\mbox{\vbox to 11pt{  \hbox{$
\int \cos^2 ax dx = \frac{x}{2}+\frac{ \sin 2ax}{4a} 
$}  }}
\\
\mbox{\vbox to 11pt{  \hbox{$
\int \cos^3 ax dx = \frac{3 \sin ax}{4a}+\frac{ \sin 3ax}{12a} 
$}  }}
\\
\mbox{\vbox to 11pt{  \hbox{$
\int \tan ax dx = -\frac{1}{a} \ln \cos ax 
$}  }}
\\
\mbox{\vbox to 11pt{  \hbox{$
\int \tan^2 ax dx = -x + \frac{1}{a} \tan ax 
$}  }}
\\
\mbox{\vbox to 11pt{  \hbox{$
\int x \cos ax dx = \frac{1}{a^2} \cos ax + \frac{x}{a} \sin ax 
$}  }}
\\
\mbox{\vbox to 11pt{  \hbox{$
\int x^2 \cos ax dx = \frac{2 x \cos ax }{a^2} + \frac{ a^2 x^2 - 2  }{a^3} \sin ax 
$}  }}
\\
\mbox{\vbox to 11pt{  \hbox{$
\int x \sin ax dx = -\frac{x \cos ax}{a} + \frac{\sin ax}{a^2} 
$}  }}
\\
\mbox{\vbox to 11pt{  \hbox{$
\int x^2 \sin ax dx =\frac{2-a^2x^2}{a^3}\cos ax +\frac{ 2 x \sin ax}{a^2} 
$}  }}
\end{footnotesize}


	\subsection{博弈游戏}
	\subsection{巴什博奕}
	\begin{enumerate}
		\item 
			只有一堆n个物品,两个人轮流从这堆物品中取物,规定每次至少取一个,最多取m个。最后取光者得胜。
		\item
			显然,如果$n=m+1$,那么由于一次最多只能取$m$个,所以,无论先取者拿走多少个,
			后取者都能够一次拿走剩余的物品,后者取胜。因此我们发现了如何取胜的法则:如果
			$n=(m+1)r+s$,(r为任意自然数,$s \leq m$),那么先取者要拿走$s$个物品,
			如果后取者拿走$k(k \leq m)$个,那么先取者再拿走$m+1-k$个,结果剩下$(m+1)(r-1)$
			个,以后保持这样的取法,那么先取者肯定获胜。总之,要保持给对手留下$(m+1)$的倍数,
			就能最后获胜。
	\end{enumerate}
\subsection{威佐夫博弈}
	\begin{enumerate}
		\item 
			有两堆各若干个物品,两个人轮流从某一堆或同时从两堆中取同样多的物品,规定每次至少取
			一个,多者不限,最后取光者得胜。
		\item
			判断一个局势$(a, b)$为奇异局势(必败态)的方法:
			$$a_k =[k (1+\sqrt{5})/2],b_k= a_k + k$$
	\end{enumerate}
\subsection{阶梯博奕}
	\begin{enumerate}
		\item
			博弈在一列阶梯上进行,每个阶梯上放着自然数个点,两个人进行阶梯博弈,
			每一步则是将一个阶梯上的若干个点(至少一个)移到前面去,最后没有点
			可以移动的人输。
		\item
			解决方法:把所有奇数阶梯看成N堆石子,做NIM。(把石子从奇数堆移动到偶数
			堆可以理解为拿走石子,就相当于几个奇数堆的石子在做Nim)
	\end{enumerate}
\subsection{图上删边游戏}
	\subsubsection{链的删边游戏}
		\begin{enumerate}
			\item
				游戏规则:对于一条链,其中一个端点是根,两人轮流删边,脱离根的部分也算被删去,最后没边可删的人输。
			\item
				做法:$sg[i] = n - dist(i) - 1$(其中$n$表示总点数,$dist(i)$表示离根的距离)
		\end{enumerate}
	\subsubsection{树的删边游戏}
		\begin{enumerate}
			\item
				游戏规则:对于一棵有根树,两人轮流删边,脱离根的部分也算被删去,没边可删的人输。
			\item
				做法:叶子结点的$sg=0$,其他节点的$sg$等于儿子结点的$sg+1$的异或和。
		\end{enumerate}
	\subsubsection{局部连通图的删边游戏}
		\begin{enumerate}
			\item
				游戏规则:在一个局部连通图上,两人轮流删边,脱离根的部分也算被删去,没边可删的人输。
				局部连通图的构图规则是,在一棵基础树上加边得到,所有形成的环保证不共用边,且只与基础树有一个公共点。
			\item
				做法:去掉所有的偶环,将所有的奇环变为长度为1的链,然后做树的删边游戏。
		\end{enumerate}

\end{document}
